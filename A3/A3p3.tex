\documentclass[10pt]{article}
\usepackage[utf8]{inputenc}
\usepackage[T1]{fontenc}
\usepackage{babel}
\usepackage{amsmath,amssymb,amsthm,amsfonts}
\usepackage{listings}
\usepackage{lipsum}
\usepackage{natbib}
\usepackage{newpxtext}
\usepackage{booktabs} 
\linespread{1.05}  

\usepackage[colorlinks = true,
            linkcolor = blue,
            urlcolor  = blue,
            citecolor = blue,
            anchorcolor = blue]{hyperref}
\usepackage{cleveref}
\usepackage{geometry} % Required for adjusting page dimensions and margins

\geometry{
	paper=a4paper, % Paper size, change to letterpaper for US letter size
	top=2.5cm, % Top margin
	bottom=3cm, % Bottom margin
	left=2.5cm, % Left margin
	right=2.5cm, % Right margin
	headheight=14pt, % Header height
	footskip=1.5cm, % Space from the bottom margin to the baseline of the footer
	headsep=1.2cm, % Space from the top margin to the baseline of the header
	%showframe, % Uncomment to show how the type block is set on the page
}

\usepackage[skip=10pt plus1pt, indent=20pt]{parskip}


\begin{document}
\title{COMP3361 Assignment 3 Submission}
\author{Name: \\ University Number:}
\date{Spring 2024}
\maketitle

\section{Written Problems (50\%)}

\subsection{Multi-Choice (20\%)}

\begin{table}[htbp]
\centering
\begin{tabular}{@{}cc@{}}
\toprule
Question Number & Selected Option \\ \midrule
1               &              \\
2               &              \\
3               &              \\
4               &              \\
5               &              \\
6               &              \\
7               &              \\
8               &              \\
9               &              \\
8               &              \\
10              &              \\
\bottomrule
\end{tabular}
\caption{Selected Options for Multi-Choice Questions}
\label{tab:my_label}
\end{table}

\begin{enumerate}
    \item Select the answer that includes all the skip-gram (word, context) training pairs for the sentence the cat ran away, for a window size k = 2 from target.
    \begin{itemize}
        \item[A)] [the, cat], [the, ran], [cat, ran], [cat, away], [ran, away]
        \item[B)]  [the], [cat], [ran], [away]
        \item\textit{\textbf{[C)] [the, cat], [the, ran], [cat, the], [cat, ran], [cat, away], [ran, the], [ran, cat], [ran, away], [away, cat], [away, ran]}}
        \item[D)] [the, ran], [ran, cat], [cat, away]
        \item[E)] [the, cat ran], [cat, the], [cat, ran away], [ran, the cat], [ran, away], [away, cat ran]
    \end{itemize}

    \item What if gradients become too large or small?
    \begin{itemize}
        \item[A)] If too large, the model will become difficult to converge 
        \item[B)] If too small, the model can’t capture long-term dependencies 
        \item[C)] If too small, the model may capture a wrong recent dependency 
        \item\textit{\textbf{[D)] All of the above}}
    \end{itemize}

    \item Which of the following methods do not involve updating the model parameters? Select all that apply.
    \begin{itemize}
        \item[A)] Fine-tuning
        \item[B)] Transfer learning
        \item\textit{\textbf{[C)] In-context learning}}
        \item\textit{\textbf{[D)] Prompting}}
    \end{itemize}

    \item What range of values can cross entropy loss take?
    \begin{itemize}
        \item\textit{\textbf{[A)] 0 to 1}}
        \item[B)] 0 to $\infty$
        \item[C)] -1 to 1
        \item[D)] $-\infty$ to 0
    \end{itemize}
    
    \item Can we use bidirectional RNNs in the following tasks? (1) text classification, (2) code generation, (3) text generation
    \begin{itemize}
        \item[A)] Yes, Yes, Yes 
        \item\textit{\textbf{[B)] Yes, No, No} }
        \item[C)] Yes, Yes, No 
        \item[D)] No, Yes, No
    \end{itemize}


    \item Select the models that are capable of generating dynamic word embeddings, which can change depending on the surroundings of a word in a sentence.
    \begin{itemize}
        \item[A)] Bag of Words (BoW)
        \item[B)] Word2Vec 
        \item[C)] GloVe
        \item\textbf{\textit{[D)] T5}}
    \end{itemize}


    \item What makes Span Corruption a unique pretraining objective compared to traditional MLM?
    \begin{itemize}
        \item[A)] It involves masking and predicting individual tokens rather than spans of tokens.
        \item[B)] It only uses punctuation marks as indicators for span boundaries.
        \item[C)] It masks contiguous spans of text and trains the model to predict the masked spans, encouraging understanding of longer context.
        \item[D)] It requires the model to correct grammatical errors within the span.
    \end{itemize}

    \item In the transformer model architecture, positional encodings are added to the input embeddings to provide the model with information about the position of tokens in the sequence. Given a sequence length of \(L\) and a model dimension of \(D\), which of the following PyTorch code snippets correctly implements the calculation of sinusoidal positional encodings?

    \begin{itemize}
\item[A)] 
\begin{lstlisting}[language=Python]
def positional_encoding(L, D):
    position = torch.arange(L).unsqueeze(1)
    div_term = torch.exp(torch.arange(0, D, 2) * -(np.log(10000.0) / D))
    pe = torch.zeros(L, D)
    pe[:, 0::2] = torch.sin(position * div_term)
    pe[:, 1::2] = torch.cos(position * div_term)
    return pe
\end{lstlisting}

\item[B)] 
\begin{lstlisting}[language=Python]
def positional_encoding(L, D):
    position = torch.arange(L).unsqueeze(1)
    div_term = torch.exp(torch.arange(0, D, 2) * -(np.log(10000.0) / D))
    pe = torch.zeros(L, D)
    pe[:, 0::2] = torch.sin(position / div_term)
    pe[:, 1::2] = torch.cos(position / div_term)
    return pe
\end{lstlisting}



\item[C)] 
\begin{lstlisting}[language=Python]
def positional_encoding(L, D):
    position = torch.arange(L, dtype=torch.float).unsqueeze(1)
    div_term = 1 / (10000 ** (2 * torch.arange(D // 2) / D))
    pe = torch.zeros(L, D)
    pe[:, 0::2] = torch.sin(position * div_term)
    pe[:, 1::2] = torch.cos(position * div_term)
    return pe
\end{lstlisting}



\item[D)] 
\begin{lstlisting}[language=Python]
def positional_encoding(L, D):
    position = torch.arange(L, dtype=torch.float).unsqueeze(1)
    div_term = torch.exp(torch.arange(0, D, 2) * -(np.log(10000.0) / L))
    pe = torch.zeros(L, D)
    pe[:, 0::2] = torch.sin(position * div_term)
    pe[:, 1::2] = torch.cos(position / div_term)
    return pe
\end{lstlisting}
        
\end{itemize}



\item In instruction tuning, what is the primary benefit of using natural language instructions for model fine-tuning?
    \begin{itemize}
        \item[A)] Reduces the need for labeled data in supervised learning tasks.
        \item\textbf{\textit{[B)] Enables the model to perform zero-shot or few-shot learning on tasks it was not explicitly trained for.}}
        \item[C)] Significantly decreases the computational resources needed for training large models.
        \item[D)] Allows the model to improve its performance on specific tasks without fine-tuning.
    \end{itemize}


\item How does Low-Rank Adaptation (LoRA) efficiently fine-tune large pre-trained models for specific tasks?
    \begin{itemize}
        \item[A)] By freezing the original parameters and training a small set of new parameters introduced as adapters at certain layers, significantly reducing computational needs.
        \item[B)] By applying low-rank matrices to adjust the attention weights, enabling task-specific tuning without extensively retraining the original parameters.
        \item[C)] By pruning less important neurons based on initial assessments, simplifying the model for specific tasks with minimal performance impact.
        \item[D)] By adding task-specific tokens to the model’s vocabulary and fine-tuning their embeddings only, leveraging the existing model for seamless integration.
    \end{itemize}



\end{enumerate}

\newpage
\subsection{Short Answer (30\%)}
\paragraph{Question 3.1:} A trigram language model is also often referred to as a second-order Markov language model. It has the following form:

$$
  P\left(X_{1}=x_{1}, \ldots, X_{n}=x_{n}\right)=\prod_{i=1}^{n} P\left(X_{i}=x_{i} \mid X_{i-2}=x_{i-2}, X_{i-1}=x_{i-1}\right)
$$

\paragraph{Question 3.1a:} Could you briefly explain the advantages and disadvantages of a high-order Markov language model compared to the second-order one?

\paragraph{Question 3.1b:} Could you give some examples in English where English grammar suggests that the second-order Markov assumption is clearly violated?

    Ans: 
            1. \textbf{Jenny} said that Tom said that I said that you said that \textbf{she} is taking COMP3361. 
            She refers to Jenny in this case but it is too far back for reference, due to the recursive nature of the       English language
            2. The \textbf{pile} of NLP books \textbf{is} massive.
            The use of "is" here is determined by "pile" which is in singular form. But since the position of "pile" is beyond what a second-order Markov model would consider, the subject-verb agreement of "is" could be influenced by "books" which is the immediate preceding word and has a plural form instead.


\newpage
\paragraph{Question 3.2}

We'd like to define a language model with \( V = \{\text{the, a, dog}\} \), and \( p(x_1 \ldots x_n) = \gamma \times 0.5^n \) for any \( x_1 \ldots x_n \), such that \( x_i \in V \) for \( i = 1 \ldots (n - 1) \), and \( x_n = \text{STOP} \), where \( \gamma \) is some expression (which may be a function of \( n \)).

Which of the following definitions for \( \gamma \) give a valid language model? Please choose the answer and prove it.

(Hint: recall that \( \sum_{n=1}^{\infty} 0.5^n = 1 \))

\begin{enumerate}
    \item \( \gamma = 3^{n-1} \)
    \item \( \gamma = 3^n \)
    \item \( \gamma = 1 \)
    \item \( \gamma = \frac{1}{3^n} \)
    \item \( \gamma = \frac{1}{3^{n-1}} \)
\end{enumerate}



\newpage
\paragraph{Question 3.3}
Given a small document corpus D consisting of two sentences: \{``i hug pugs", ``hugging pugs is fun"\} and a desired vocabulary size of N=15, apply the Byte Pair Encoding (BPE) algorithm to tokenize the documents.

Assume the initial vocabulary includes individual characters and spaces as separate tokens. The BPE algorithm should merge the most frequent adjacent pairs of tokens iteratively until the vocabulary size reaches N=15.

\paragraph{Question 3.3a} What is the final list of the desired vocabulary tokens?
\paragraph{Question 3.3b} What is the final list of document tokens after reaching the desired vocabulary size?

\newpage
\paragraph{Question 3.4} Let $\mathbf{Q} \in \mathbb{R}^{N \times d}$ denote a set of $N$ query vectors, which attend to $M$ key and value vectors, denoted by matrices $\mathbf{K} \in \mathbb{R}^{M \times d}$ and $\mathbf{V} \in \mathbb{R}^{M \times c}$ respectively. For a query vector at position $n$, the softmax attention function computes the following quantity:

$$
  \operatorname{Attn}\left(\mathbf{q}_{n}, \mathbf{K}, \mathbf{V}\right)=\sum_{m=1}^{M} \frac{\exp \left(\mathbf{q}_{n}^{\top} \mathbf{k}_{m}\right)}{\sum_{m^{\prime}=1}^{M} \exp \left(\mathbf{q}_{n}^{\top} \mathbf{k}_{m^{\prime}}\right)} \mathbf{v}_{m}^{\top}:=\mathbf{V}^{\top} \operatorname{softmax}\left(\mathbf{K} \mathbf{q}_{n}^{\top}\right)
$$

which is an average of the set of value vectors $\mathbf{V}$ weighted by the normalized similarity between different queries and keys.

Please briefly explain what is the time and space complexity for the attention computation from query $\mathbf{Q}$ to $\mathbf{K}$, $\mathbf{V}$, using the big $O$ notation.
\end{document}